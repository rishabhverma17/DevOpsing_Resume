%%%%%%%%%%%%%%%%%
% This is an example CV created using altacv.cls (v1.1.5, 1 December 2018) written by
% LianTze Lim (liantze@gmail.com), based on the
% Cv created by BusinessInsider at http://www.businessinsider.my/a-sample-resume-for-marissa-mayer-2016-7/?r=US&IR=T
%
%% It may be distributed and/or modified under the
%% conditions of the LaTeX Project Public License, either version 1.3
%% of this license or (at your option) any later version.
%% The latest version of this license is in
%%    http://www.latex-project.org/lppl.txt
%% and version 1.3 or later is part of all distributions of LaTeX
%% version 2003/12/01 or later.
%%%%%%%%%%%%%%%%

%% If you are using \orcid or academicons
%% icons, make sure you have the academicons
%% option here, and compile with XeLaTeX
%% or LuaLaTeX.
% \documentclass[10pt,a4paper,academicons]{altacv}

%% Use the "normalphoto" option if you want a normal photo instead of cropped to a circle
% \documentclass[10pt,a4paper,normalphoto]{altacv}

\documentclass[10pt,a4paper,ragged2e]{altacv}

%% AltaCV uses the fontawesome and academicon fonts
%% and packages.
%% See texdoc.net/pkg/fontawecome and http://texdoc.net/pkg/academicons for full list of symbols. You MUST compile with XeLaTeX or LuaLaTeX if you want to use academicons.

% Change the page layout if you need to
\geometry{left=1cm,right=9cm,marginparwidth=6.8cm,marginparsep=1.2cm,top=1.25cm,bottom=1.25cm}

% Change the font if you want to, depending on whether
% you're using pdflatex or xelatex/lualatex
\ifxetexorluatex
  % If using xelatex or lualatex:
  \setmainfont{Carlito}
\else
  % If using pdflatex:
  \usepackage[utf8]{inputenc}
  \usepackage[T1]{fontenc}
  \usepackage[default]{lato}
\fi

% Change the colours if you want to
\definecolor{VividPurple}{HTML}{3E0097}
\definecolor{SlateGrey}{HTML}{2E2E2E}
\definecolor{LightGrey}{HTML}{666666}
\colorlet{heading}{VividPurple}
\colorlet{accent}{VividPurple}
\colorlet{emphasis}{SlateGrey}
\colorlet{body}{LightGrey}

% Change the bullets for itemize and rating marker
% for \cvskill if you want to
\renewcommand{\itemmarker}{{\small\textbullet}}
\renewcommand{\ratingmarker}{\faCircle}

%% sample.bib contains your publications
\addbibresource{sample.bib}

\begin{document}
\name{Rishabh Verma}
\tagline{Software Development Engineer}
% Cropped to square from https://en.wikipedia.org/wiki/Marissa_Mayer#/media/File:Marissa_Mayer_May_2014_(cropped).jpg, CC-BY 2.0
\photo{2.5cm}{img}
\personalinfo{%
    \email{rishabhverma17@gmail.com}
    %\twitter{@marissamayer}
    \linkedin{linkedin.com/in/rishabhverma1}
    \homepage{https://www.rishabhverma.in}
    %\phone{+91-9958783667}
    \github{github.com/rishabhverma17} 
    \location{Bangalore, India}
    % I'm just making this up though.
%   \orcid{orcid.org/0000-0000-0000-0000} % Obviously making this up too. If you want to use this field (and also other academicons symbols), add "academicons" option to \documentclass{altacv}
}

%% Make the header extend all the way to the right, if you want.
\begin{fullwidth}
\makecvheader
\end{fullwidth}

%% Depending on your tastes, you may want to make fonts of itemize environments slightly smaller
\AtBeginEnvironment{itemize}{\small}

%% Provide the file name containing the sidebar contents as an optional parameter to \cvsection.
%% You can always just use \marginpar{...} if you do
%% not need to align the top of the contents to any
%% \cvsection title in the "main" bar.
\cvsection[page1sidebar]{Summary}
\textbf{Problem solver with experience in developing low latency, highly scalable, fault-tolerant, distributed backend services and evolving the architecture for performance and scalability.Experience of Object-Oriented Programming, Object-Oriented Design, Event Streaming, Data Structures and Algorithms, Prevalent Design Patterns, Caching, NoSQL, RDBMS. }

\cvsection{EXPERIENCE}
\cvevent{Software Development Engineer}
{Yatra Online Pvt Ltd (Yatra.com)}{Jan 2020 -- Ongoing}{Bangalore, India}{Online Travel Company}
\begin{itemize}
\item Focused on developing low latency and highly scalable distributed backend services for Yatra Hotels.
\item Designed a central logger service for all external microservices metric logging. This service is made agnostic and easy to integrate with other microservices.
\item Working on improving the existing Dedup content pipeline for Auto Suggest Microservice.
\end{itemize}
\divider

\cvevent{Software Development Engineer}
{Delhi Integrated Multi Modal Transit System (DIMTS)}{Jan 2019 -- Jan 2020}{New Delhi, India}{Public Transit}
\begin{itemize}
\item Designed DMRC and DIMTS Reconciliation system which reduced reconciliation time by 80\% and also eliminated manual effort.
\item Developed tool for automated recovery of trips and shifts corrupted due to operational issues of ETM machine.
\item Redesigned ETM Challan module so that ETM machines can be moved more than once a day, thus reducing ETM Movement cost by 30\%.
\end{itemize}
\divider

\cvevent{Software Engineer}{Aperta Limited}{Aug 2016 -- Jan 2019}{Coimbatore, India}{FinTech}
\begin{itemize}
\item Developed Bank’s in-house utility which interacts with Core banking system (CBS) and Cheque truncation system (CTS).
\item Reduced the time by 75\% to process 70,000 to 1, 00,000 instruments from 16 seconds to less than 4 seconds by redesigning the algorithm.
\item Developed CHM utility which reduced time by 30\% to extract banking software's business rules sent from RBI and update all Nodal branch systems remotely over network.
\end{itemize}
\divider

\cvevent{Software Engineer Intern}{Defence Research and Development Organisation (D.R.D.O)}{June 2015 -- July 2015}{New Delhi, India}{}
\begin{itemize}
\item Designed a virtual reality environment to perform experiment for learning cognitive enhancement by navigation training. Results indicated that soldiers trained with survey perspective view performed at least 30\% more accurate than those with route perspective.
\end{itemize}

\clearpage

% \cvsection[page2sidebar]{Publications}
% 
% \nocite{*}
% 
% \printbibliography[heading=pubtype,title={\printinfo{\faBook}{Books}},type=book]
% 
% \divider
% 
% \printbibliography[heading=pubtype,title={\printinfo{\faFileTextO}{Journal Articles}}, type=article]
% 
% \divider
% 
% \printbibliography[heading=pubtype,title={\printinfo{\faGroup}{Conference Proceedings}},type=inproceedings]
% 
%% If the NEXT page doesn't start with a \cvsection but you'd
%% still like to add a sidebar, then use this command on THIS
%% page to add it. The optional argument lets you pull up the
%% sidebar a bit so that it looks aligned with the top of the
%% main column.
% \addnextpagesidebar[-1ex]{page3sidebar}


\end{document}
